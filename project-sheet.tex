\documentclass[8pt]{article}

\usepackage[utf8]{inputenc}
\usepackage[ngerman]{babel}
\usepackage{amsmath}
\usepackage{nicefrac}
\usepackage{color}
\usepackage[hidelinks]{hyperref}

\usepackage[a4paper,top=2cm,left=4cm,right=4cm,bottom=2cm,marginparwidth=5cm]{geometry}
\usepackage{marginnote}
\usepackage[automark]{scrpage2}
\pagestyle{scrheadings}

\newcommand{\fatnote}[1]{\marginnote{\textbf{#1}}}
\newcommand{\leftnote}[1]{\reversemarginpar\fatnote{#1}}
\newcommand{\rightnote}[1]{\normalmarginpar\fatnote{#1}}

\newcommand{\defin}[1]{\noindent #1\vspace{0.3cm}}
\newcommand{\ldefin}[2]{\leftnote{#1}\defin{#2}}
\newcommand{\rdefin}[2]{\rightnote{#1}\defin{#2}}

\newcommand{\todo}[1]{\textcolor{red}{\textbf{TODO}: #1}}

\newcommand{\coursename}{\@empty}
\newcommand{\groupno}{\@empty}

\newcommand{\course}[2]{\renewcommand{\coursename}{#1}\renewcommand{\groupno}{#2}}
\newcommand{\beginsheet}{\clearscrheadfoot\ihead[]{Kurs: \coursename}\ohead[]{Gruppe \groupno}\ofoot[]{\pagemark}\ifoot[]{Dieses Dokument ist Teil der Dokumentation}}

\newcommand{\email}[1]{\href{mailto:#1}{#1}}


%%%%%%%%%%%%%%%%%%%%%%%%%%%%%%%%%%%%%%%%%%%%%%%%%%%%%%%%%%%%%%%%%%%%%%%%%%%%%%%%%%%%%%
%% Oberhalb dieses Blocks nichts ändern
%%%%%%%%%%%%%%%%%%%%%%%%%%%%%%%%%%%%%%%%%%%%%%%%%%%%%%%%%%%%%%%%%%%%%%%%%%%%%%%%%%%%%%


%% TODO: Hier die fehlende Gruppennummer einfügen
\course{Lisp Kurs -- Roboterprogrammierung in Lisp}{\todo{Nr. einfügen}}


\begin{document}

\beginsheet

\section*{Projekt-Übersicht}
Stand: \today\\[0.25cm]
Mitglieder:\\[0.25cm]
%% TODO: Hier die Mitglieder samt E-Mail-Adressen eintragen
\begin{tabular}{|p{0.5\columnwidth}|p{0.5\columnwidth}|}
  \hline
  \textbf{Name} & \textbf{E-Mail} \\
  \hline
  \hline
  Maik Schünemann & \email{person1@example.com} \\
  \hline
  Mitglied 2 & \email{person2@example.com} \\
  \hline
  Mitglied 3 & \email{person3@example.com} \\
  \hline
\end{tabular}
\vspace{0.25cm}\\
Betreuer:\\[0.25cm]
\begin{tabular}{|p{0.5\columnwidth}|p{0.5\columnwidth}|}
  \hline
  \textbf{Name} & \textbf{E-Mail} \\
  \hline
  \hline
  Jan Winkler & \email{jwinkler@uni-bremen.de} \\
  \hline
\end{tabular}
\vspace{0.25cm}\\
Projekt-Repository: \url{...} \todo{URL zum Source-Code einfügen}

\subsection*{Thema und Problemstellung}
%% TODO: Hier das Thema und die Problemstellung beschreiben
Es existiert ein "Spielfeld", welches mit Hindernissen und einem "gelben Fleck" bestückt ist.

Man setzt den Roboter an einer beliebigen Stelle in das Feld.
Er soll den gelben Fleck finden und dann auf dem schnellsten Weg zurückbringen




\subsection*{Angestrebte Problemlösung}
%% TODO: Hier die angestrebte Problemlösung beschreiben
Der Roboter versucht nun durch Tiefensuche den gelben Fleck zu finden und, wenn er diesen gefunden hat aus dem ihm bekannten Teil des Feldes den kürzesten Weg zu berechnen und zurück fahren.

\end{document}
